\chapter{Conclusion and Future Work}
\label{conclusion}

The main objective of this thesis was to develop a theoretical framework in order to further our understanding of the increase in scale in cooperation in knowledge intensive production processes that we have witnessed during the last decades. The central approach that guided this theory building effort on large scale cooperation is to focus on meso level social processes in order to bridge the gap between that most comon approaches to cooperation in the literature: a more classical macro level approach focused on collective visions and plans, and a micro level approach focused on the dynamics of individual dyadic interactions. This meso level approach is focused on the structural dimension of cooperation in these production processes, that is, our focus is on the patterns of relations that direct producers establish in knowledge intensive production processes.

Focusing on these patterns of relations, we proposed that the subgroups of individuals that are formed in them are a key element to understand and explain how coordiantion problems are managed in large scale cooperation in big knowledge intensive production processes. The formation and dissolution of these subgroups, their high level of turnover in their individual composition, their role in shaping individual contributions to the whole production processes, and their role in maintaining individuals attached to the project, are some of the elements explored in this thesis that enabled us to build a theoretical framwork that bridges the gap between macro and micro levels theoretical approaches to cooperation.

For us, the structural approach to cooperation, and to any social process, is synonim of a network approach because, at its core, network theory is a relational approach that allows a quantitative rigorous way to model patterns of relation among individuals, and thus enables our theories and accounts of social processes to go beyond the reduccionist approach to understand social interactions as pure dyadic interactions between individuals.

An important part of the research effort developed for this thesis has focused on the problem of how to determine cohesive subgroups in social networks. The problem is not new, and a lot of work has been done in this direction, but as usual in the social sciences, the methods available to researches are far behind the theoretical insights when we want to exploit the increasing availability of data that should support the empirical work associated with the theories that we develop.

Concretelly we found that the techniques available to identify the cohesive subgroups most interesting from a theoretical point of view ---that is those based on the graph theory concept of node connectivity--- were not able to handle the scale of the cooperation networks in which we were interested. We developed new and original heurictic techniques that allowed us to analyze networks one order of magnitude bigger than the ones that the classical techinques were able to handle. We put a lot of efford to make sure that these new techciques that we developed are available to other researches, and thus we contributed our implementation to a popular free software programs for the analysis of networks.

All these methodological effords have paid out in theoretical terms, as we were able to test empirically a new network model that we proposed to further our understanding how collaborative communities work. The model that we named ``cohesive small world'' is a good fit to describe the patterns of relations of the two big and mature FOSS projectes that we have analyzed in the empirical part of this thesis.

However, it is not clear that this theorical model fits all cooperation networks of FOSS projects. Because the empirical analysis was a case study of two successful projects we cannot determine if other FOSS projects also fit nicely in it. Thus what we presented is more an existence proof than a empirical contrastation of our theorical model.

The next steps in our research agenda will have to try to expand the kind of empirical analysis that we presented in this thesis in order to include both sucessful and unsuccessful large scale cooperation production process beyond in order to assess up to which point the model can explain the achivements and continuity in time of FOSS projects.

Moreover, it is necessary to include in our empirical analysis other knowledge intensive production processes beyond FOSS projects in order to make sure that the cohesive small world model is a solid theoretical framework to explain and understand the large scale cooperation that emerged in the last decades in knowledge intensive production processes.

We are specialy interested in analyze cooperation among scientists but we feel that our model is also useful to analyze any kind of large scale knowledge intensive procudction process, such as the ones that are developed inside big capitalist corporations and state sponsored endeavors, such as spacial exploration. 
