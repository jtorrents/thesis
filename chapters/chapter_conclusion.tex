\chapter{Conclusion and Future Work}
\label{conclusion}

The main objective of this thesis was to develop a theoretical framework in order to further our understanding on how large scale cooperation works in knowledge intensive production processes in the context of new organizational forms, such as FOSS projects. The central approach that guided this theory building effort on large scale cooperation is to focus on meso level social processes in order to bridge the gap between that most common approaches to cooperation in the literature: a more classical macro level approach focused on collective visions, shared values, and authority forms; and a micro level approach focused on the dynamics of individual dyadic cooperative interactions. This meso level approach is focused on the structural dimension of cooperation in these production processes, that is, my focus is on the patterns of relations that direct producers establish in knowledge intensive production processes.

Focusing on these patterns of relations, I proposed that the subgroups of individuals that are formed in cooperation networks are a key element to understand and explain how coordination problems are managed in large scale cooperation in knowledge intensive production processes with high individual turn over. The formation and dissolution of these subgroups, their high level of turnover in their individual composition, their role in shaping individual contributions to the whole production process, and their role in maintaining individuals attached to the project, are some of the elements explored in this thesis that enabled me to build a theoretical framework that bridges the gap between macro and micro levels theoretical approaches to cooperation.

I centered the development of this new theoretical framework on what I named the Cohesive Small World model. A network model grounded on two well established theoretical network models: the Small World model \citep{watts:1998} and the Structural Cohesion model \citep{white:2001, moody:2003}. These two models are not mutually exclusive. The family of networks that fit in the intersection of both models ---what I call Cohesive Small Worlds--- exhibit consistent topological patterns, that is, they have the same structure. These patterns, I argue, provide the scaffolding for the emergence of collaborative communities \citep{adler:2006} and enable effective large scale cooperation.

On the one hand, the generation of trust and congruent values among heterogeneous individuals are fostered by structurally cohesive groups in the connectivity hierarchy of cooperation networks because individuals embedded in these structures are able to compare independent perspectives on each other through a variety of paths that flow through distinct sets of intermediaries, which provides multiple independent sources of information about each other. Thus, the perception of an individual embedded in such structures of the other members of the group to whom she is not directly linked is filtered by the perception of a variety of others whom she trusts because is directly linked to them. This mediated perception of the group generates trust at a global scale. On the other hand, the existence of dense local clusters connected between them by relative short paths allows successful cooperation among heterogeneous individuals with common interests and, at the same time, fosters the flow of information between these clusters.

The structural approach to cooperation, and to any social process, is for me synonym of a network approach because, at its core, the network approach is a relational approach that allows a quantitative rigorous way to model patterns of relation among individuals, that is, to model social structures. And thus enables our theories and accounts of social processes to go beyond the reductionist approach to understand social interactions as pure dyadic interactions between individuals without losing the quantitative rigor usually associated with the methodological individualism approach to social interaction.

An important part of the research effort developed for this thesis has focused on the problem of how to determine cohesive subgroups in social networks as prescribed by the structural cohesion model. The problem is not new, and a lot of work has been done in this direction, but as usual in the social sciences, the methods available to researches are far behind the theoretical insights when we want to exploit the increasing availability of data that should support the empirical work associated with the theories that we develop.

I extended theoretically the structural cohesion model by considering not only plain node connectivity, which is the minimum number of nodes that must be removed in order to disconnect a network, but also the average node connectivity of networks and its cohesive groups, which is the number of nodes that, on average, must be removed to disconnect an arbitrary pair of nodes in the network. Taking into account average connectivity allows a more granular conception of structural cohesion, and I show in the empirical analysis of cooperation networks how this approach leads to useful implications in empirical research.

I also developed heuristics to compute the $k$-components structure, along with the average node connectivity for each $k$-component, based on the fast approximation to compute node independent paths \citep{white:2001b}. These heuristics allow for the computing of the approximate value of group cohesion for moderately large networks, along with all the hierarchical structure of connectivity levels, in a reasonable time frame. I showed that these heuristics can be applied to networks at least one order of magnitude bigger than the ones manageable by the exact algorithm proposed by \citet{moody:2003}. To ensure reproducibility and facilitate diffusion of these heuristics I contributed an implementation to a popular Python software package for the analysis of complex networks: NetworkX \citep{hagberg:2008}. See appendix \ref{publications} for the documentation and source code at NetworkX github repository. I believe that providing detailed implementation is critical to ensure reproducibility, but often these details are black-boxed, some times because of proprietary software restrictions or authors' reluctance to share their work.

All these methodological efforts have paid out in theoretical terms, as I was able to test empirically the new network model that I proposed to further our understanding how large scale cooperation works in the context of Collaborative Communities. The model that I named ``Cohesive Small World'' is a good fit to describe the cooperation networks ---that is, the patterns of relations between direct producers--- of the two big and mature FOSS projects that I have analyzed in the empirical part of this thesis: the CPython reference implementation of the Python programming language, and the Debian operating system.

The analysis presented in chapter \ref{cohesive_small_world} shows that the cooperation networks of both Debian and Python projects can be modeled using the proposed Cohesive Small World model. It is interesting to note that they also show significative differences because Debian cooperation networks lean more to the Small World end of the model, while Python cooperation networks lean more towards the Structural Cohesion end of the model. The difference in terms of modularity of the product that they are building ---an Operating System versus a Programming language--- impacts their respective production processes. Debian's subgroups tend to work more independently from each other than Python's subgroups, as shown by the fact that Debian cooperation networks exhibit a higher degree of smallworldiness; while Python's networks are more structurally cohesive as shown by their sharper and steep connectivity hierarchy.

It's necessary to note that the structural analysis presented in this thesis does not cover all the important elements needed for an in depth analysis of what enables successful cooperation in general. Things like the individual characteristics of the people that cooperate, the complementarity of their skills in relation to the task at hand, the emergence of leadership in the context of teams, are indeed also important for enabling successful cooperation. But the point that I tried to make in this thesis is that the structural dimension of cooperation is at least equally important and has been a lot less explored by theoretical accounts of cooperation. 

To further the empirical analysis of this thesis, in chapter \ref{contributions} I explored the dynamic dimension of the connectivity hierarchies that emerge on the cooperation networks of the Python and Debian projects. I defined cooperation networks as the patterns of relations among developers established while contributing to the project. The dynamic analysis, in this case, is not only a longitudinal account of the changes in the connectivity hierarchy through time, but also the analysis of the pace of renewal of individuals in the positions defined by this hierarchy.

I show that the Cohesive Small World model is a solid theoretical framework to define cohesive groups in cooperation networks. The nested structure of $k$-components nicely captures the hierarchy in the patterns of relations that individual contributors establish when working together. This hierarchy, on the one hand, reflects the empirically well established fact that in FOSS projects only a small fraction of the developers account for most of the contributions. And, on the other hand, refutes the naive views of early academic accounts that characterized FOSS projects as a flat hierarchy of peers in which every individual does more or less the same. 

I also show that the position of individual developers in the connectivity hierarchy of the cooperation networks impacts significantly, on the one hand, on the volume of contributions that an individual does to the project. And, on the other hand, the median active life of an individual in the project. I argue that the latter is a better way to analyze robustness of FOSS projects than the classical random and targeted attacks that has been used to asses robustness in other kinds of networks and that it's the standard approach to assess robustness in the network literature \citep{albert:2000}.

I argue that the connectivity structure of collaborative communities' cooperation networks can be characterized as an open elite, where the top levels of this hierarchy are filled with new individuals at a high pace. This feature is key for understanding the mechanisms and dynamics that make FOSS communities able to develop long term projects, with high individual turnover, and yet achieve high impact and coherent results as a result of large scale cooperation. Therefore, I can conclude that cooperation in FOSS communities has a structural dimension because membership in cohesive groups that emerge from the cooperation networks ---the repeated patterns of relations that the direct producers establish in the production process--- has an important and statistically significative impact on both the volume of individual contributions, and on the median active life of developers in the projects under analysis.

It is worth noting that the high rate of turn over in the top positions of the connectivity hierarchy of cooperation networks ---which I argue is a key mechanism for enabling large scale cooperation--- is only possible assuming that the knowledge necessary to perform the tasks in the FOSS project is highly socialized. As I pointed out in the introduction, the concept of socialization of the production used by Marx refers to the process of replacing tacit knowledge generated by small groups in local contexts by knowledge that is explicitly codified and disseminated at a global level. In Marx's own words ``only socialized labour [...] is capable of applying the general products of human development, such as mathematics, to the immediate process of production'' \citep[1024]{marx:1990}. Therefore large scale cooperation is only possible in highly socialized production processes.

It is easier to analyze this kind of highly socialized production processes in FOSS projects than in capitalist corporations because the latter have to balance the need of enhancing large scale cooperation trough knowledge socialization with the pressures of the profit imperative. In Marxian terms we could say that in FOSS projects there is no contradiction between the progressive socialization of the production and the private appropriation of the result of the production process typical of capitalist production relations. This is because the result of the production process in FOSS projects is Free Software, which in many senses is quite similar to mathematics, and thus can be considered a general product of human development, as Marx puts it in the previous quote.

The empirical analysis presented in this thesis also sheds light over the findings of recent empirical analysis of individual contributions in collaborative communities \citep{shaw:2014}, where the authors find that only a small fraction of participants are the ones that contribute most of its contents, and they thus propose that some form of the ``iron law of oligarchy'' \citep{michels:1915} might be in play. They however do not analyze longitudinally if these people are the same throughout the history of the project. I suspect that they might not be the same people, and thus that a constant renewal of the people that contribute the most, such the one described here for the Python project, might also be in play in those projects. 

However, it is not clear that the theoretical model that I proposed fits all cooperation networks of Collaborative Communities, or even cooperation networks of all FOSS projects. Because the empirical analysis presented in this thesis was a case study of two successful projects aimed to develop a theoretical framework, I cannot determine if other FOSS projects also fit nicely in it. Thus what I presented is more an existence proof than a empirical test of my proposed theoretical model.

The next steps in my research agenda will have to expand the kind of empirical analysis that I presented in this thesis to include both successful and unsuccessful large scale cooperation production process in order to assess up to which point the Cohesive Small World model can explain the achievements and continuity in time of FOSS projects and other collaborative communities.

Thus, it is necessary to include in the empirical analysis of the Cohesive Small World model other knowledge intensive production processes beyond FOSS projects in order to make sure that this model is a solid theoretical framework to explain and understand how large scale cooperation works in knowledge intensive production processes.

I'm especially interested in analyzing cooperation among scientists, but I feel that the Cohesive Small World model can also be useful to analyze any kind of large scale knowledge intensive production process with a high degree of cooperation, such as the ones that are developed inside big capitalist corporations, and on state sponsored large scale scientific and technical endeavors, such as space exploration.
