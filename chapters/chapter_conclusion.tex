\chapter{Conclusion and Future Work}
\label{conclusion}

The main objective of this thesis was to develop a theoretical framework in order to further our understanding of the increase in scale of cooperation in knowledge intensive production processes that we have witnessed during the last decades \citet*{uzzi:2007a}. The central approach that guided this theory building effort on large scale cooperation is to focus on meso level social processes in order to bridge the gap between that most comon approaches to cooperation in the literature: a more classical macro level approach focused on collective visions and plans, and a micro level approach focused on the dynamics of individual dyadic interactions. This meso level approach is focused on the structural dimension of cooperation in these production processes, that is, our focus is on the patterns of relations that direct producers establish in knowledge intensive production processes.

Focusing on these patterns of relations, we proposed that the subgroups of individuals that are formed in them are a key element to understand and explain how coordiantion problems are managed in large scale cooperation in big knowledge intensive production processes. The formation and dissolution of these subgroups, their high level of turnover in their individual composition, their role in shaping individual contributions to the whole production process, and their role in maintaining individuals attached to the project, are some of the elements explored in this thesis that enabled us to build a theoretical framwork that bridges the gap between macro and micro levels theoretical approaches to cooperation.

We centered the developement of this new theoretical framwork on what we named the cohesiove small world model. A network model grounded on two well stablished theoretical network models: the small world model \citep{watts:1998} and the structural cohesion model \citep{white:2001, moody:2003}. These two models are not mutually exclusive. The family of networks that fit in the intersection of both models ---what we call cohesive small worlds--- exhibit consistent topological patterns. These patterns, we argue, provide the scaffolding for the emergence of collaborative communities \citep{adler: 2006}. On the one hand, the generation of trust and congruent values among heterogeneous individuals are fostered by structurally cohesive groups in the connectivity hierarchy of cooperation networks that play a key role in amplifying the effects of social interactions trough relatively long paths. On the other hand, the existence of highly connected local clusters allows successful cooperation among heterogeneous individuals with common interests.

For us, the structural approach to cooperation, and to any social process, is synonim of a network approach because, at its core, network theory is a relational approach that allows a quantitative rigorous way to model patterns of relation among individuals, and thus enables our theories and accounts of social processes to go beyond the reduccionist approach to understand social interactions as pure dyadic interactions between individuals.

An important part of the research effort developed for this thesis has focused on the problem of how to determine cohesive subgroups in social networks as prescribed by the structural cohesion model. The problem is not new, and a lot of work has been done in this direction, but as usual in the social sciences, the methods available to researches are far behind the theoretical insights when we want to exploit the increasing availability of data that should support the empirical work associated with the theories that we develop.

We extended theoretically the structural cohesion model by considering not only plain node connectivity, which is the minimum number of nodes that must be removed in order to disconnect a network, but also the average node connectivity of networks and its cohesive groups, which is the number of nodes that, on average, must be removed to disconnect an arbitrary pair of nodes in the network. Taking into account average connectivity allows a more granular conception of structural cohesion, and we show in our empirical analysis of cooperation networks how this approach leads to useful implications in empirical research.

We also developed heuristics to compute the $k$-components structure, along with the average node connectivity for each $k$-component, based on the fast approximation to compute node independent paths \citep{white:2001b}. These heuristics allow for the computing of the approximate value of group cohesion for moderately large networks, along with all the hierarchical structure of connectivity levels, in a reasonable time frame. We showed that these heuristics can be applied to networks at least one order of magnitude bigger than the ones manageable by the exact algorithm proposed by \citet{moody:2003}. To ensure reproducibility and facilitate diffusion of these heuristics we contributed an implementation to a popular Python software package for the analysis of complex networks: NetworkX \citep{hagberg:2008}. We believe that providing detailed implementation is critical to ensure reproducibility, but often these details are black-boxed, some times because of proprietary software restrictions or authors' reluctance to share their work.

All these methodological effords have paid out in theoretical terms, as we were able to test empirically the new network model that we proposed to further our understanding how collaborative communities work. The model that we named ``cohesive small world'' is a good fit to describe the patterns of relations of the two big and mature FOSS projectes that we have analyzed in the empirical part of this thesis: the CPython reference implementation of the Python programming language, and the Debain operating system.

The analysis presented in chapter \ref{cohesive_small_world} shows that the cooperation networks of both Debain and Python projects can be modelled using our proposed cohesive small world model. It is intersting to note that they also show significative differences because Debain cooperation networks basculate more to the small world end of the model, while Python cooperation networks basculate more towards the structural cohesion end of the model. The difference in terms of modularity of the product that they are building ---an Operating System versus a Programming language--- impacts their respective production processes. Debian's subgroups tend to work more independently from each other than Python's subgroups, as shown by the fact that Debian cooperation networks exhibit a higher degree of smallworldiness; while Python's networks are more structurally cohesive as shown by their sharper and steep connectecty hierarchy.

To further our empirical analysis, in chapter \ref{contributions} we explored the dynamical dimension of the connectivity hierarchies that emerge on the cooperation networks of the Python and Debian projects. We defined cooperation networks as the patterns of relations among developers established while contributing to the project. The dynamic analysis, in this case, is not only a longitudinal account of the changes in the hierarchy through time, but also the analysis of the pace of renewal of individuals in the positions defined by the hierarchy.

We showed that the cohesive small world model is a solid theoretical framework to define cohesive groups in cooperation networks. The nested structure of $k$-components nicely captures the hierarchy in the patterns of relations that individual contributors establish when working together. This hierarchy, on the one hand, reflects the empirically well established fact that in FOSS projects only a small fraction of the developers account for most of the contributions. And, on the other hand, refutes the naive views of early academic accounts that characterized FOSS projects as a flat hierarchy of peers in which every individual does more or less the same.

We also showed that the position of individual developers in the connectivity hierarchy of the cooperation networks impacts significantly, on the one hand, on the volume of contributions that an individual does to the project. And, on the other hand, the median life time of developers in the project. We argue that the latter is a better way to analyze robustness of FOSS projects than the classical random and targeted attacks that has been used to asses robustness in other kinds of networks.

We argue that the connectivity structure of collaborative communities' cooperation networks can be characterized as an open elite, where the top levels of this hierarchy are filled with new individuals at a high pace. This feature is key for understanding the mechanisms and dynamics that make FOSS communities able to develop long term projects, with high individual turnover, and yet achieve high impact and coherent results. Therefore, we can conclude that cooperation in FOSS communities has a structural dimension because membership in cohesive groups that emerge from the cooperation networks ---the repeated patterns of relations that the direct producers establish in the production process--- has an important and statistically significative impact on both the volume of individual contributions, and on the median active life of developers in the projects under analysis.

However, it is not clear that the theorical model taht we proposed fits all cooperation networks of collaborative communities, or even cooperation networks of all FOSS projects. Because the empirical analysis presented in this thesis was a case study of two successful projects aimed to develop a theoretical framework, we cannot determine if other FOSS projects also fit nicely in it. Thus what we presented is more an existence proof than a empirical contrastation of our theorical model.

The next steps in our research agenda will have to try to expand the kind of empirical analysis that we presented in this thesis in order to include both sucessful and unsuccessful large scale cooperation production process in order to assess up to which point the model can explain the achivements and continuity in time of FOSS projects and other collaborative communities.

Thus, it is necessary to include in the empirical analysis of the cohsive small world model other knowledge intensive production processes beyond FOSS projects in order to make sure that this model is a solid theoretical framework to explain and understand the large scale cooperation that emerged in the last decades in knowledge intensive production processes.

We are specialy interested in analyze cooperation among scientists but we feel that our model can also be useful to analyze any kind of large scale knowledge intensive procudction process, such as the ones that are developed inside big capitalist corporations, and on state sponsored large scale scientific and technical endeavors, such as space exploration.
