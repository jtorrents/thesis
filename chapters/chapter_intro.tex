\chapter{Cooperation as a Key Social Mechanism of Socialization}

Robert Merton popularized Isaac Newton's quote ``if I have seen further it is by standing on the shoulders of giants''. This quote highlight, on the one hand, the cumulative nature of the scientific knowledge and, on the other hand, implies that the contribution to knowledge production by highly talented individuals ---the giants--- is far more important than the contribution of ordinary individuals. If we look at the history of science, the names of Archimedes, Galilei, Newton, Euler, Darwin, Einstein and a few others shine strongly. It is certainly true that without their contribution to knowledge, our understanding of the universe would be far less deep and sharp. But, nowadays, have giants such a key role in the production of knowledge?  

The last half of twentieth century has witnessed a key shift in the production process of knowledge. Based on works on science and enginering, social sciences, arts and humanities and Patents, \citet*{uzzi:2007a} show that until 1950s the likelihood that an important ---ie wildly cited--- paper or invention was developed by a single author was bigger than it was developed by a team. But this trend has experimented a shift in the last four decades. The rising importance of collective research and teamwork is illustrated by the fact that top cited papers in all those disciplines are mostly created by teams in 2000s.

The fact that, in the twentyfirst century, the most important discoveries and innovations in science and technology are not anymore the result of the work of very talented individuals working alone is well established but untheorized. The aim of this research is to propose a theoretical explanation of this social process along with the principal social mechanisms that drive this trend. I will argue that the social process that can explain this empirical evidence is cooperation.

The use of the concept socialization in this research draws on the Marxian tradition. The common use of this term in mainstream social science is somewhat different but related. As Paul Adler put it, ``in recent Marxist writings as in political science more generally, socialization refers to the transfer of ownership from the private to the public sphere. In psychology, socialization is commonly construed as the process whereby people new to a culture internalize its knowledge, norms and values. Marx's use was broader than either and encompasses both.'' \citep[1320]{adler:2007}. Then Adler cite a relevant passage of Capital: 

\begin{quote}
The social productive forces of labour, or the productive forces of directly social, socialized (i.e. collective) labour come into being through cooperation, division of labour within the workshop, the use of machinery, and in general, the transformation of production by the conscious use of the sciences, of mechanics, chemistry, etc. for specific ends, technology, etc. and similarly, through the enormous increase of scale corresponding to such developments (for it is only socialized labour that is capable of applying the general products of human development, such as mathematics, to the immediate process of production; and conversely, progress in these sciences presupposes a certain level of material production). \citep[1024]{marx:1990}
\end{quote}

Following Marx's argument, we could say that what was once only achievable by a gifted mind working alone is now within reach of ordinary minds through cooperation, division of labour, the use of machinery and by the conscious use of the science and technology. The aim of this research is to focus on cooperation as a key social mechanism that allow organizations achieve high impact in the development of complex technology and in the production of knowledge.

\section{A Meso Level Approach to Cooperation}

The central topic of this research project is understand and explain under which conditions and through which social mechanisms large scale cooperation emerges in an open organizational environment. Therefore, one of the main theoretical challenges is conceptualize the social process of cooperation. There are two main approaches to conceptualize cooperation in the literature: as an atomic process in which cooperation is produced between two individuals and, on the other hand, as a macro-level phenomenon in which the center of analysis is the collectivity or group.

Karl Marx is a classical exponent of the latter approach. Accordint to him, two key dimensions of cooperation are the shared goal that guide the social process and its collective nature: ``\emph{[w]hen numerous workers work together side by side in accordance with a plan, whether in the same process, or in different but connected processes, this form of labour is called co-operation}'' \citep[443]{marx:1990}. Marx highlights that the principal characteristic of cooperation is that the final result of coordinated action is much more than the sum of the individual actions. In Marx words, ``\emph{[...] the sum total of the mechanical forces exerted by isolated workers differs from the social force that is developed when many hands co-operate in the same undivided operation [...] [n]ot only do we have here an increase in the productive power of the individual, by means of co-operation, but the creation of a new productive power, which is intrinsically a collective one.}'' \citep[443]{marx:1990}.

On the other hand, there are the dyadic approaches to cooperation \citep{axelrod1981,axelrod:1997}. Those approaches are based on the assumption that interactions between pairs of individuals occur on a probabilistic basis. From this standpoint, models are developed based on the concept of an evolutionary strategy on the context of an iterated Prisoner's Dilemma game. Based on deductions from such a model and agent-based simulations, those approaches show how cooperation based on reciprocity can get started in an asocial world, can thrive while interacting with a wide range of other strategies and can resist invasion once fully established \citep{axelrod1981}.

Those approaches to cooperation are very useful to conceptualize cooperation in an evolutionary and interspices scenario. On one hand, the payoffs of iterated Prisoner's Dilemma are not assumed to be commensurable, and on the other hand, the players (ie, organisms) do not need brain to employ a strategy. So this model can explain not only interactions between two bacteria and two primates (for example \emph{homo sapiens}), but it can also explain interactions between a colony of bacteria and a primate serving as a host \citep[211]{axelrod1981}. Thus, these models of cooperation explain the emergence of relevant mutualist biological relations as symbiosis.

The classical dyadic approach developed by Axelrod was the basis on which further refinements have been build. Watts, in his seminal book on small world networks \citep{watts:1999}, tested the effects of the small world topology on dyadic interactions finding that the initial configuration of strategies of the nodes of networks are critical in the evolution of strategies ---cooperation or defection--- without been able to establish a strong relation between topology and cooperative strategy. In the same stream of research, the analysis of emergence of role differentiation in an hierarchical network environment, based upon Spatial Prisioner's Dilemma, showed that leaders ---nodes with a large payoff who are imitated by an important fraction of the population--- play an essential role in sustaining a cooperative regime \citep*{eguiluz:2005}.

All those approaches have in common their dyadic nature, their grounding in agent-based simulations and their reductionist approach. We can conceptualize reductionism in this context as the assumption that macro processes ---such as mutualist biological relations or cooperation in a productive process developed in an organization--- must be understood only in terms of the actions and relations of the individuals involved in the process. I will argue, based in the Marxian approach to cooperation, that we must differentiate between cooperation strategies developed in a conscient level as a production process and mutualist biological relations, where payoffs of iterated Prisoner's Dilemma are not assumed to be commensurable and the players does not need brain to employ a strategy.

The aim of this research is to bridge the gap between macro-level and micro-level approaches to cooperation by focusing on meso-level mechanisms, which until recently have recieved little attention in the theoretical debate focused on the two extremes highlighted above. I argue that focusing on the structural dimension of cooperation, that is, the patterns of relations between the individuals that participate in production processes. This prespective shows that between the dyadic interactions among individuals, and the shared goals and visions that guide large organizations and groups, there are subgroups of individuals that play a key role in allowing the increase of scale of cooperation that we have witnessed during the last decades.        

\section{Free and Open Source Software as the perfect example}

The development of an open source project ---such as Debian or Python--- can be conceptualized as a social system build on the top of the complex technological system of Free and Open Source Software (FOSS). This technological system is composed of all the free software that is written and released. This system has experimented an impressive increment of scale in the past two decades \citep{ghosh:2006}. Approximately two thirds of the existing free software is developed by individual programmers working collaboratively, 15\% by for-profit companies and 20\% other organizations (academic, social ,...). According to the calculations presented in \citet{ghosh:2006}, if capitalist companies wanted to reproduce internally the production of free software in use, it would cost approximately 12 billion euros\footnote{That is, 12,000,000,000 euros.} that would be used primarily to pay the workforce. The code base of free software has doubled every 18-24 months over the last years. According to estimates of the authors of this report, this trend will continue over the next few years. This code base can be quantified, at least, in 131,000 person-years of work. This work has been developed mainly by unpaid programmers.

The impressive momentum gained by FOSS, exemplified by an outstanding increment of scale, is only possible by the emergence of complex social systems on top of it that, in turn, feed this increment of scale. The FOSS phenomenon have attracted some research efforts from different scientific fields during last years. The main focus have been in four related areas: a) its microfundaments or individual incentives, ie why individuals decide to involve in FOSS development despite the fact that only by narrow self-interest and instrumental rationality it should be better for them to free-ride other's efforts \citep{hars:2002,lakhani:2003,hertel:2003,weber:2004,roberts:2006}; b) innovation and intellectual property policy, ie how community managed or corporate sponsored FOSS projects are able to innovate in order to solve complex technical problems and freely reveal those innovations without appropriating private returns from selling the software \citep{kogut:2001,hippel:2001,hippel:2003,vonKrogh2003,omahony:2003,west:2003,lerner:2005,hargrave:2006,west:2008}; c) development methods, ie how FOSS communities manage coordination and complexity developing large software systems, this stream of research have been mainly addressed from a software enginering perspective \citep{godfrey:2000,feller:2000,mockus:2002,koch:2002,weber:2004,maccormack:2006}; d) organization and governance, ie how communities producing public goods govern themselves, this stream of research addresses the classic problem of how individuals coordinate their actions in order to achieve collective outcomes \citep{ljungberg:2000,omahony:2007,west:2008}.

This research will be centred in the intersection of areas c) and d) described above. The aim is to analyze the development methods of FOSS from the perspective of a knowledge based production process in order to address the classical sociological problem of social organization of production and its dependencies and relations with political organization and governance, in the sense of how individuals coordinate their actions in order to achieve collective outcomes. The major difference of this perspective from the works based on software engineering that previously addressed the development methods issue is that the empirical source ---software development--- is not an objective in itself; it is a proxy to analyze a potentially new form of organization of a complex knowledge based production process that do not rely mainly on market or hierarchy mechanisms. I will argue that the socialization process is a key theoretical element to understand the mechanisms and the outcomes of this new production paradigm.

This research is a case study, and its principal aim is to develop theoretical insights in order to build a framework that allow us to understand and explain both the new characteristics of this production process as well as the elements of continuity with capitalist production processes. Thus, in this research I will analyze cooperation in the Debian and Python projects, in order to show that the social structure resulting from the cooperation among their individual participants is characterized by the formation of subgroups that form the scafolding that allows the bast increment in scale that empirical research has reported in the last decades.
