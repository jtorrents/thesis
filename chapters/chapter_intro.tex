\chapter{Theoretical approaches to Cooperation}
\label{intro}

Robert Merton popularized Isaac Newton's quote ``if I have seen further it is by standing on the shoulders of giants''. This quote highlight, on the one hand, the cumulative nature of the scientific knowledge and, on the other hand, implies that the contribution to knowledge production by highly talented individuals ---the giants--- is far more important than the contribution of ordinary individuals. If we look at the history of science, the names of Archimedes, Galilei, Newton, Euler, Darwin, Einstein and a few others shine strongly. It is certainly true that without their contribution to knowledge, our understanding of the universe would be far less deep and sharp. But, nowadays, have giants such a key role in the production of knowledge?  

The last half of twentieth century has witnessed a key shift in the production process of knowledge. Based on works on science and engineering, social sciences, arts and humanities and Patents, \citet*{uzzi:2007a} show that until 1950s the likelihood that an important ---ie wildly cited--- paper or invention was developed by a single author was bigger than it was developed by a team. But this trend has experimented a shift in the last four decades. The rising importance of collective research and cooperation is illustrated by the fact that top cited papers in all those disciplines are mostly created by teams in 2000s.

The fact that, in the twenty first century, the most important discoveries and innovations in science and technology are not anymore the result of the work of very talented individuals working alone but the result of cooperation is well established but untheorized. We argue that a key element of the social processes that help explain this empirical evidence is the increase in scale of cooperation as a key social mechanism of socialization. The aim of this research is to propose a theoretical explanation of how large scale cooperation works in the context of knowledge intensive production processes.

We approach the concept socialization in this research drawing on the Marxian tradition. The common use of this concept in mainstream social science is somewhat different but related. As Paul Adler put it, ``in recent Marxist writings as in political science more generally, socialization refers to the transfer of ownership from the private to the public sphere. In psychology, socialization is commonly construed as the process whereby people new to a culture internalize its knowledge, norms and values. Marx's use was broader than either and encompasses both.'' \citep[1320]{adler:2007}. Then Adler cite a relevant passage of Capital: 

\begin{quote}
The social productive forces of labour, or the productive forces of directly social, socialized (i.e. collective) labour come into being through cooperation, division of labour within the workshop, the use of machinery, and in general, the transformation of production by the conscious use of the sciences, of mechanics, chemistry, etc. for specific ends, technology, etc. and similarly, through the enormous increase of scale corresponding to such developments (for it is only socialized labour that is capable of applying the general products of human development, such as mathematics, to the immediate process of production; and conversely, progress in these sciences presupposes a certain level of material production). \citep[1024]{marx:1990}
\end{quote}

Following Marx's argument, we could say that what was once only achievable by a gifted mind working alone is now within reach of ordinary minds through cooperation, division of labour, the use of machinery and by the conscious use of the science and technology. The aim of this research is to focus on cooperation as a key social mechanism that allow organizations achieve high impact in the development of complex technology and in the production of knowledge.

\section{A Meso Level Approach to Cooperation}

The central topic of this thesis is to understand and explain under which conditions and through which social mechanisms large scale cooperation operates in an open organizational environment. Therefore, one of the main theoretical challenges is conceptualize the social process of cooperation. There are two main approaches to conceptualize cooperation in the literature: as an atomic process in which cooperation is produced between two individuals and, on the other hand, as a macro level phenomenon in which the center of analysis is the collective or group.

Karl Marx is a classical exponent of the latter approach. According to him, two key dimensions of cooperation are the shared goal that guide the social process and its collective nature: ``\emph{[w]hen numerous workers work together side by side in accordance with a plan, whether in the same process, or in different but connected processes, this form of labour is called co-operation}'' \citep[443]{marx:1990}. Marx highlights that the principal characteristic of cooperation is that the final result of coordinated action is much more than the sum of the individual actions. In Marx words, ``\emph{[...] the sum total of the mechanical forces exerted by isolated workers differs from the social force that is developed when many hands co-operate in the same undivided operation [...] [n]ot only do we have here an increase in the productive power of the individual, by means of co-operation, but the creation of a new productive power, which is intrinsically a collective one.}'' \citep[443]{marx:1990}. According to Marx, cooperation constitutes the starting point of capitalist production \citep[439]{marx:1990}, not only in historical terms, but also conceptually.

More recently, new macro approaches to cooperation have been developed in order to deal with the challenging problem of understanding and explaining how knowledge is produced and disseminated in the context of production processes that depend on effective innovation \cite{adler:2015}. Based on the Marxian tradition, \citep{adler:2006} introduce the concept of collaborative communities to make sense of novel organizational forms ---both inside and outside large capitalist corporations--- strongly grounded on large scale cooperation which were defying the traditional dichotomy between hierarchy and market \citep{coase:1937,williamson:1975} as coordinating mechanisms for production processes. Collaborative communities are characterized by conscious cooperation, high individual interdependence, trust, shared values and a value-rational basis for legitimate authority \citep{adler:2006,adler:2008}. Thus, this macro level approach to explain large scale cooperation focuses on values, norms, generalized trust, and authority forms as the key elements that enable large scale cooperation inside capitalist corporations, and on new emerging open organizational environments.

According to \citet{adler:2015}, recent literature in organization studies \citep{omahony:2011} has highlighted the role of community \citep{tonnies:1974} as a critical precondition for innovation in large scale production processes. Adler proposes an alternative reading of Marx's theory of the capitalist production process where community ---in the form of what Marx calls the ``collective worker''--- is an essential feature the labour process, even under antagonistic capitalist employment conditions \citep[446]{adler:2015}. The need to create use-values in the labour process makes large scale cooperation essential in complex production processes, but the need to obtain exchange-value in the valorization process of the Capital, that is, the capitalist firm's profitability imperative, undermines and thwarts the full potential of large scale cooperation. Adler argues that the emergence of a new collaborative form of community in recent decades can be understood as communism developing in the heard of capitalist production processes.

On the other hand, there are the dyadic approaches to cooperation \citep{axelrod1981,axelrod:1997}. Those approaches are based on the assumption that interactions between pairs of individuals occur on a probabilistic basis. From this standpoint, models are developed based on the concept of an evolutionary strategy on the context of an iterated Prisoner's Dilemma game. Based on deductions from such a model and agent-based simulations, those approaches show how cooperation based on reciprocity can get started in an asocial world, can thrive while interacting with a wide range of other strategies, and can resist invasion once fully established \citep{axelrod1981}.

Those approaches to cooperation are very useful to conceptualize cooperation in an evolutionary and interspecies scenario. On one hand, the payoffs of iterated Prisoner's Dilemma are not assumed to be commensurable, and on the other hand, the players (i.e. organisms) do not need brain to employ a strategy. So this model can explain not only interactions between two bacteria and two primates (for example \emph{homo sapiens}), but it can also explain interactions between a colony of bacteria and a primate serving as a host \citep[211]{axelrod1981}. Thus, these models of cooperation explain the emergence of relevant mutualist biological relations as symbiosis.

The classical dyadic approach developed by Axelrod was the basis on which further refinements have been build. Watts, in his seminal book on small world networks \citep{watts:1999}, tested the effects of the small world topology on dyadic interactions finding that the initial configuration of strategies of the nodes of networks are critical in the evolution of strategies ---cooperation or defection--- without been able to establish a strong relation between topology and cooperative strategy. In the same stream of research, the analysis of emergence of role differentiation in an hierarchical network environment, based upon Spatial Prisioner's Dilemma, showed that leaders ---nodes with a large payoff who are imitated by an important fraction of the population--- play an essential role in sustaining a cooperative regime \citep*{eguiluz:2005}.

All those approaches have in common their dyadic nature, their grounding in agent-based simulations, and their reductionist approach. We can conceptualize reductionism in this context as the assumption that macro processes ---such as mutualist biological relations or cooperation in knowledge intensive production process--- must be understood only in terms of the actions and relations of the individuals involved in the process. I argue however that we must differentiate between cooperation strategies developed in a conscient level ---such as production processes--- and mutualist biological relations, where payoffs of iterated Prisoner's Dilemma are not assumed to be commensurable and the players do not even need a brain to employ a cooperation strategy. The reductionist approach to cooperation is extremely general and thus can highlight commonalities between social and biological cooperation processes, but this generality also hinders its power to successfully modeling all the nuances of complex social process, such as the knowledge intensive production processes that are the focus of this thesis.    

The aim of this research is to bridge the gap between macro level and micro level approaches to cooperation by focusing on meso level mechanisms, which until recently have received little attention in the theoretical debate focused on the two extremes highlighted above. I argue that a meso level approach has to focus on the structural dimension of cooperation, that is, the patterns of relations between the individuals that participate in production processes. This perspective shows that between the atomic dyadic interactions among individuals, and the shared goals, values, and visions that guide large organizations and groups, there are subgroups of individuals that play a key role in effectively enabling large scale cooperation to work, as we have witnessed during the last decades.

In order to conceptualize and clearly define how to analyze the patterns of relations that individual participants establish in knowledge intensive production processes, I use the concept of cooperation networks as a kind of social network. A social network is composed of social actors and their interactions. Social network literature \citep*{wasserman:1994, scott:2011}, and more generally what has been recently named network science \citep*{newman:2003, newman:2006, newman:2010, easley:2010}, provide methods for analyzing the structure of these networks along with the processes that are developed in, and by, these networks. In the concrete case of knowledge intensive production processes, the objects that are the result of the production process can also be included as part of the cooperation networks, as I will show in the empirical analysis part of this thesis. This approach allows for an in deep analysis of the different levels of individual contributions to the whole cooperation process, which has been one of the empirical puzzles that has driven the literature on cooperation in new open organizational forms.

This network approach to cooperation links with the micro approaches to cooperation in that it focuses on the dyadic interaction of actors, but instead of stopping at the dyadic level, I will analyze the groups that emerge from the global patterns of relations. That is, groups of actors that are more intensely interconnected among them than with the rest of the actors of the network. I argue that the formation and dissolution of these groups through time, and their individual composition and turn over are key elements to explain how large scale cooperation works in practice. These meso level mechanisms link with the macro level approach to cooperation because they help explain how generalized trust, shared values, and non-despotic forms of authority can emerge and be maintained through time in large organizations and on informal groups and communities.

Obtaining detailed data about knowledge intensive production processes is a challenging endeavor, specially from big capitalist corporations, because there are no public records and their internal processes are implicitly considered a secret. However, in order to build a theoretical framework it is imperative to have detailed information on how the production processes actually work. This is the main reason I focus on open organizational environments, namely Free and Open Source Software communities, as a source of empirical data. The production processes in this context are developed mainly through the Internet, and the details and electronic records of these production processes are public and freely available.

\section{Free and Open Source Software as a case study}

Free Software, broadly defined, is computer software that allows users to run, copy, distribute, study, change and improve it. Thus, what defines Free Software is not its price but the freedoms that their users enjoy. Richard Stallman aptly summarizes it by saying that ``to understand the concept, you should think of ``free'' as in ``free speech'', not as in ``free beer''.'' \citep[3]{stallman:2002}. In the late 1990s the term Open Source Software was used to refer to this same concept in a less ideological and more business friendly way. Though there are important philosophical differences between the two names used to refer to this kind of software \citep[75]{stallman:2002b} for the objectives of this research can be used interchangeably. This is also the case for most research on the phenomenon, and a common practice in the literature is to refer to it as Free and Open Source Software (FOSS). We follow the same convention in this thesis.

The case studies that are the focus of the empirical analysis for this thesis are the Debian project, which releases a complete operating system, and the Python project, a general propose programming language. The development of an open source project ---such as Debian or Python--- can be conceptualized as a social system build on the top of the complex technological system of Free and Open Source Software (FOSS). This technological system is composed of all the free software that is written and released.

FOSS has experimented an impressive increment of scale in the past two decades \citep{ghosh:2006}. Approximately two thirds of the existing free software is developed by individual programmers working collaboratively, 15\% by for-profit companies and 20\% other organizations (academic, social ,...). According to the calculations presented in \citet{ghosh:2006}, if capitalist companies wanted to reproduce internally the production of free software in use, it would cost approximately 12 billion euros\footnote{We refer here to english billions, that is 12,000,000,000 euros.} that would be used primarily to pay the workforce. The code base of free software has doubled every 18-24 months over the last years. According to estimates of the authors of this report, this trend will continue over the next few years. This code base can be quantified, at least, in 131,000 person-years of work. This work has been developed mainly by unpaid programmers.

The impressive momentum gained by FOSS, exemplified by an outstanding increment of scale, is only possible by the emergence of complex social systems on top of it that, in turn, feed this increment of scale. The FOSS phenomenon have attracted some research efforts from different scientific fields during last years. The main focus have been in four related areas: a) its microfundaments or individual incentives, ie why individuals decide to involve in FOSS development despite the fact that only by narrow self-interest and instrumental rationality it should be better for them to free-ride other's efforts \citep{hars:2002,lakhani:2003,hertel:2003,weber:2004,roberts:2006}; b) innovation and intellectual property policy, ie how community managed or corporate sponsored FOSS projects are able to innovate in order to solve complex technical problems and freely reveal those innovations without appropriating private returns from selling the software \citep{kogut:2001,hippel:2001,hippel:2003,vonKrogh2003,omahony:2003,west:2003,lerner:2005,hargrave:2006,west:2008}; c) development methods, ie how FOSS communities manage coordination and complexity developing large software systems, this stream of research have been mainly addressed from a software engineering perspective \citep{godfrey:2000,feller:2000,mockus:2002,koch:2002,weber:2004,maccormack:2006}; d) organization and governance, ie how communities producing public goods govern themselves, this stream of research addresses the classic problem of how individuals coordinate their actions in order to achieve collective outcomes \citep{ljungberg:2000,omahony:2007,west:2008}.

This research will be centered in the intersection of areas c) and d) described above. The aim is to analyze the development methods of FOSS from the perspective of a knowledge based production process in order to address the classical sociological problem of social organization of production and its dependencies and relations with political organization and governance, in the sense of how individuals coordinate their actions in order to achieve collective outcomes. The major difference of this perspective from the works based on software engineering that previously addressed the development methods issue is that the empirical source ---software development--- is not an objective in itself; it is a proxy to analyze large scale cooperation in the context of new organizational forms, FOSS projects, that develop a complex knowledge based production process that do not rely mainly on market or hierarchy mechanisms in order to guide individual decisions and actions. I argue that focusing on the patterns of relations of direct producers ---that is, focusing on cooperation networks--- we can analyze the meso level mechanisms that enable and foster large scale cooperation.

The empirical part of this thesis is thus a case study, and its principal aim is to develop theoretical insights in order to build a framework that allow us to understand and explain both the new characteristics of this production process as well as the elements of continuity with capitalist production processes. Thus, this thesis will analyze cooperation in the Debian and Python projects, in order to show that the social structure resulting from the cooperation among their individual participants is characterized by the formation of subgroups that form the scaffolding that allows large scale of cooperation to work, as empirical research has reported in the last decades in the production of knowledge, in general, and the development of FOSS, in particular.

\section{Objectives of this thesis}

The main research question of this thesis can be succinctly stated as: How does large scale cooperation work in knowledge intensive and technically complex production processes developed in new organizational environments, such as FOSS projects, where loosely coupled individuals that rarely meet face to face have to coordinate through internet in order to produce world class software products.

In more detail, the objectives of this thesis can be classified as:

\begin{description}

\item[Methodological] Contributions to the state of the art techniques of social network analysis.

\begin{enumerate}

\item Design and implement a fast approximation algorithm for computing structural cohesion in order to be able to analyze large cooperation networks.

\item Design and implement a new visualization technique for structural cohesion analysis.

\end{enumerate}

\item[Theoretical] Contributions to the sociological literature on structural cohesion, cooperation, and collaborative communities.

\begin{enumerate}

\item Extend theoretically the structural cohesion model by introducing the consideration of average node connectivity on top of the plain node connectivity. 

\item Propose a new network model for modeling the patterns of relations between direct producers in collaborative communities, named ``Cohesive Small World'', that is based on two well known network models: the Small World model and the Structural Cohesion model. 

\item Explore the meso level mechanisms that are developed in the nested cohesive subgroups structure of cooperation networks and their impact in the diffusion of information, the convergence on common values and shared goals, and specially in how generalized trust is maintained between individuals that rarely meet face to face, and how this enables the resilience of the whole cooperation network to individual turnover.

\item Contribute to the collaborative communities literature from a structural perspective by showing the key role of meso level structures, such as subgroups of individuals in the connectivity hierarchy of cooperation networks, in their effectiveness in the production and diffusion of knowledge through large scale cooperation. 

\end{enumerate}

\item[Empirical] Contributions to the empirical analysis of FOSS projects. 

\begin{enumerate}

\item The literature on FOSS, especially from software engineering and computer science, have stressed that only a small fraction of the developers is doing most of the work. I focus on who actually are these developers by analyzing the network structure that emerges form cooperation among participants in the community. 

\item The empirical analysis presented in this thesis shows that the developers that contribute the most are in the higher levels of the connectivity hierarchy of cooperation networks.

\item The developers that are in the higher levels of the connectivity hierarchy of cooperation networks change significantly through the history of the project. This is an indication of an open elite dynamic, where new developers are able to access the top levels of the hierarchy.

\item This is an essential meso level mechanism that help explain, on the one hand, the long term survival of community based projects, and on the other hand, how large scale cooperation works in the face of high individual turn over in the context of new organizational forms.

\end{enumerate}

\item[Practical] Contributions of general interest beyond academia.

\begin{enumerate}

\item Make widely available the implementation of the new approximation algorithm for structural cohesion presented on this thesis by contributing a suitable implementation to NetworkX \citep{hagberg:2008}, a popular Python Free Software project for the analysis of the structure and dynamics of complex networks.

\end{enumerate}

\end{description}

%\subsection{Outline of the contents of this thesis}
