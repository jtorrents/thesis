\selectlanguage{english}%
\begin{abstract}
The last half of twentieth century has witnessed a key shift in the production process of knowledge: the most important discoveries and innovations in science and technology are not anymore the result of the work of very talented individuals working alone, but the result of cooperation and teamwork. The remarkable increase in scale of cooperation in knowledge intensive production processes has renewed the interest in analyzing the mechanisms by which large scale cooperation emerges and thrives.

The two main theoretical approaches to cooperation are, on the one hand, a micro approach that considers cooperation as an atomic process in which cooperation is produced between two individuals and, on the other hand, as a macro level phenomenon in which the center of analysis is the collectively or group. The aim of this research is to bridge the gap between macro level and micro level approaches to cooperation by focusing on meso level mechanisms, which until recently have received little attention in the theoretical debate. We argue that a meso level approach has to focus on the structural dimension of cooperation, that is, the patterns of relations between the individuals that participate in production processes. This perspective shows that between the dyadic interactions among individuals, and the shared goals and visions that guide large organizations and groups, there are subgroups of individuals that play a key role in allowing the increase of scale of cooperation that we have witnessed during the last decades.

This research focuses on the case study of two large, mature, and successful Free and Open Source Software (FOSS) projects ---the Debian operating system and the Python programming language--- in order to build a structural theoretical framework that helps explain and understand how large scale cooperation works. We present a network model, that we name Cohesive Small World, which is based on two well established network models: the Small World model and the Structural Cohesion model. We propose that these two models are not mutually exclusive. The family of networks that fit in the intersection of both models exhibit consistent topological patterns. These patterns, we argue, provide the scaffolding for the emergence of collaborative communities, such as FOSS projects. On the one hand, the generation of trust and congruent values among heterogeneous individuals are fostered by structurally cohesive groups in the connectivity hierarchy of cooperation networks that play a key role in amplifying the effects of social interactions through relatively long paths. On the other hand, the existence of highly connected local clusters allows successful cooperation among heterogeneous individuals with common interests.

We developed heuristics to compute the $k$-components structure, along with the average node connectivity for each $k$-component. These heuristics allow to compute the approximate value of group cohesion for moderately large networks, along with all the hierarchical structure of connectivity levels, in a reasonable time frame. We show that these heuristics can be applied to networks at least one order of magnitude bigger than the ones manageable by only algorithm available until now. We test empirically the new network model that we proposed to further our understanding how cooperation in collaborative communities works. We find that the model that we named ``cohesive small world'' is a good fit to describe the cooperation patterns of the two big and mature FOSS projects that we have analyze in the empirical part of this thesis.

To further our empirical analysis, we explore the dynamical dimension of the connectivity hierarchies that emerge on the cooperation networks of the Python and Debian projects. We defined cooperation networks as the patterns of relations among developers established while contributing to the project. The dynamic analysis that we present is not only a longitudinal account of the changes in the hierarchy through time, but also the analysis of the pace of renewal of individuals in the positions defined by the hierarchy. We show that the cohesive small world model is a solid theoretical framework to define cohesive groups in cooperation networks. The nested structure of $k$-components nicely captures the hierarchy in the patterns of relations that individual contributors establish when working together. This hierarchy, on the one hand, reflects the empirically well established fact that in FOSS projects only a small fraction of the developers account for most of the contributions. And, on the other hand, refutes the naive views of early academic accounts that characterized FOSS projects as a flat hierarchy of peers in which every individual does more or less the same. 

We also show that the position of individual developers in the connectivity hierarchy of the cooperation networks impacts significantly, on the one hand, on the volume of contributions that an individual does to the project. And, on the other hand, the median life time of developers in the project. We argue that the latter is a better way to analyze robustness of FOSS projects than the classical random and targeted attacks that has been used to assess robustness in other kinds of networks.

We argue that the connectivity structure of collaborative communities' cooperation networks can be characterized as an open elite, where the top levels of this hierarchy are filled with new individuals at a high pace. This feature is key for understanding the mechanisms and dynamics that make FOSS communities able to develop long term projects, with high individual turnover, and yet achieve high impact and coherent results. We conclude that cooperation in FOSS communities has a structural dimension because membership in cohesive groups that emerge from the cooperation networks has an important and statistically significative impact on both the volume of individual contributions, and on the median active life of developers in the projects under analysis.
\end{abstract}

\newpage

\selectlanguage{catalan}%
\begin{abstract}
L'última meitat del segle XX ha estat testimoni d'un canvi fonamental en el procés de producció de coneixement: els descobriments més importants i les innovacions en ciència i tecnologia no són el resultat de la tasca de persones amb molt talent que treballant soles, sinó que són el resultat de processos de cooperació i de treball en equip. El notable augment de l'escala de la cooperació en els processos de producció intensius en coneixement ha renovat l'interès en l'anàlisi dels mecanismes pels quals emergeix i prospera la cooperació a gran escala.

Els dos principals enfocaments teòrics sobre la cooperació són, d'una banda, un enfocament micro que considera que la cooperació com un procés atòmic en el qual l'interès es centra en com es produeix cooperació entre dues persones i, d'altra banda, com un fenomen a nivell macro en el qual el centre de l'anàlisi és el grup com a col·lectivitat. L'objectiu d'aquesta recerca és acostar posicions entre l'enfoc macro i el micro sobre la cooperació tot centrant-nos en els mecanismes a nivell meso, que fins fa poc han rebut poca atenció en el debat teòric. Argumentem que un enfocament de nivell meso ha de centrar-se en la dimensió estructural de la cooperació, és a dir, els patrons de relacions entre els individus que participen en els processos de producció. Aquesta perspectiva mostra que entre les interaccions diàdiques entre els individus, i els grans objectius i visions compartides que guien les grans organitzacions i grups, hi ha subgrups d'individus que tenen un paper clau en generar l'augment d'escala de la cooperació del que hem estat testimonis en les últimes dècades.

Aquesta recerca es centra en l'estudi de cas de dos projectes de programari lliure (FOSS en anglès) ---el sistema operatiu Debian i el llenguatge de programació Python--- per tal de construir un marc teòric estructural que ens ajudi a explicar i entendre com funciona la cooperació gran escala. Presentem un model de xarxa, que anomenem ``Cohesive Small World'', que es basa en dos models teòrics ben establertes: el model ``Small World'' i el model de cohesió estructural. Proposem que aquests dos models no són mútuament excloents. La família de xarxes que s'ajusten a la intersecció de tots dos models mostren patrons topològics consistents. Aquests patrons proporcionen els fonaments per al sorgiment de comunitats de col·laboració, com ara projectes de programari lliure. D'una banda, grups estructuralment cohesius en la jerarquia de connectivitat de les xarxes de cooperació generen confiança i valors compartits entre individus heterogenis que juguen un paper clau en l'amplificació dels efectes de les interaccions socials a través d'enllaços no directes entre individus. D'altra banda, l'existència d'agrupacions locals d'individus altament connectats permet la cooperació entre individus heterogenis però amb interessos comuns.

En la part metodològica de la tesi, hem desenvolupat heurístiques per a calcular l'estructura de $k$-components de les xarxes de cooperació. Aquestes heurístiques permeten calcular el valor aproximat de la cohesió de grups per a xarxes moderadament grans, juntament amb tota l'estructura jeràrquica dels nivells de connectivitat, en un temps raonable. Demostrem com aquestes heurístiques poden ser aplicades a xarxes d'almenys un ordre de magnitud més grans que les que podia assumir l'únic algoritme disponible fins ara. Amb l'ajuda d'aquestes heurístiques posem a prova empíricament el nou model que proposem per tal de millorar la nostra comprensió de com funciona la cooperació en les comunitats de col·laboració. L'anàlisi empírica demostra que el model estructural que proposem s'ajusta als patrons de cooperació que observem en els projectes de programari lliure que analitzem en la part empírica de la tesi.

L'anàlisi empírica d'aquesta tesi explora la dimensió dinàmica de les jerarquies de connectivitat que sorgeixen en les xarxes de cooperació dels projectes de Python i Debian. Definim les xarxes de cooperació com els patrons de relació entre les persones que participen en els processos productius dels projectes analitzats. L'anàlisi dinàmic que presentem no és només una anàlisi longitudinal dels canvis en la jerarquia a través del temps, sinó també l'anàlisi del ritme de renovació dels individus en les posicions definides per aquesta jerarquia. Demostrem que el model estructural que proposem és un marc teòric sòlid per definir grups cohesius en les xarxes de cooperació. L'estructura imbricada de $k$-components defineix la jerarquia dels patrons de relacions que estableixen els individuals al treballar conjuntament. Aquesta jerarquia, d'una banda, reflecteix el fet empíricament ben establert que en projectes de programari lliure només una petita part dels desenvolupadors contribueix la major part de la feina feta en cada projecte. I, d'altra banda, refuta les opinions ingènues de primers relats acadèmics que caracteritzen els projectes de programari lliure com una jerarquia plana de persones en la qual cada individu fa més o menys el mateix.

L'anàlisi empírica d'aquesta test també mostra que la posició dels desenvolupadors individuals en la jerarquia de la connectivitat de les xarxes de cooperació impacta significativament, d'una banda, el volum de les contribucions individuals al projecte. I, d'altra banda, el temps de vida mitjana dels desenvolupadors en el projecte, entesa com el temps que de mitjana una persona és participant activa en el projecte.

Argumentem que l'estructura de connectivitat de xarxes de cooperació de les comunitats de col·laboració pot caracteritzar-se com una elit oberta, on els nivells més alts d'aquesta jerarquia es renoven amb nous individus a un ritme alt. Aquesta característica és clau per entendre els mecanismes i dinàmiques que fan que les comunitats de programari lliure siguin capacess de desenvolupar projectes a llarg termini, amb un alt volum de renovació individual, i no obstant això, aconsegueixin uns resultats coherents. Finalment arribem a la conclusió que la cooperació en les comunitats de programari lliure té una dimensió estructural ja que la pertinença a grups cohesius que sorgeixen en les xarxes de cooperació té un impacte important i estadísticament significatiu tant en el volum de les contribucions individuals com en la vida activa mitjana dels desenvolupadors en els projectes que analitzem.
\end{abstract}
\selectlanguage{english}%

