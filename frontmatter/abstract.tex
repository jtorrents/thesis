\selectlanguage{english}%
\begin{abstract}
The last half of twentieth century has witnessed a key shift in the production process of knowledge: the most important discoveries and innovations in science and technology are not anymore the result of the work of very talented individuals working alone, but the result of cooperation and teamwork. The remarkable increase in scale of cooperation in knowledge intensive production processes has renewed the interest in analyzing the mechanisms by which large scale cooperation emerges and thrives.

The two main theoretical approaches to cooperation are, on the one hand, a micro approach that considers cooperation as an atomic process in which cooperation is produced between two individuals and, on the other hand, as a macro level phenomenon in which the center of analysis is the collectively or group. The aim of this research is to bridge the gap between macro level and micro level approaches to cooperation by focusing on meso level mechanisms, which until recently have received little attention in the theoretical debate. I argue that a meso level approach has to focus on the structural dimension of cooperation, that is, the patterns of relations between the individuals that participate in production processes, what I call cooperation networks. This perspective shows that between the dyadic interactions among individuals, and the shared goals and values that guide large organizations and groups, there are subgroups of individuals that play a key role in enabling the kind of large scale cooperation that we have witnessed during the last decades.

This research focuses on the case study of two large, mature, and successful Free and Open Source Software (FOSS) projects ---the Debian operating system and the Python programming language--- in order to build a structural theoretical framework that helps explain and understand how large scale cooperation works. I present a network model, that I name Cohesive Small World, which is based on two well established network models: the Small World model and the Structural Cohesion model. I propose that these two models are not mutually exclusive. The family of networks that fit in the intersection of both models exhibit consistent structural patterns. These patterns, I argue, provide the scaffolding for the emergence of collaborative communities, such as FOSS projects, and enable and foster effective large scale cooperation.

On the one hand, the generation of trust and congruent values among heterogeneous individuals are fostered by structurally cohesive groups in the connectivity hierarchy of cooperation networks because individuals embedded in these structures are able to compare independent perspectives on each other through a variety of paths that flow through distinct sets of intermediaries, which provides multiple independent sources of information about each other. Thus, the perception of an individual embedded in such structures of the other members of the group to whom she is not directly linked is filtered by the perception of a variety of others whom she trusts because is directly linked to them. This mediated perception of the group generates trust at a global scale. On the other hand, the existence of dense local clusters connected between them by relative short paths allows successful cooperation among heterogeneous individuals with common interests and, at the same time, fosters the flow of information between these clusters preventing the local clusters to be trapped in echo chambers of like minded collaborators.

I developed heuristics to compute the $k$-components structure, along with the average node connectivity for each $k$-component. These heuristics allow to compute the approximate value of group cohesion for moderately large networks, along with all the hierarchical structure of connectivity levels, in a reasonable time frame. I show that these heuristics can be applied to networks at least one order of magnitude bigger than the ones manageable by the only algorithm available until now. I test empirically the new network model that I proposed to further our understanding how cooperation in collaborative communities works. I find that the model that I named ``Cohesive Small World'' is a good fit to describe the cooperation patterns of the two big and mature FOSS projects that I analyze in the empirical part of this thesis.

To further the empirical analysis, I explore the dynamic dimension of the connectivity hierarchies that emerge on the cooperation networks of the Python and Debian projects. I defined cooperation networks as the patterns of relations among developers established while contributing to the project. The dynamic analysis that I present is not only a longitudinal account of the changes in the hierarchy through time, but also the analysis of the pace of renewal of individuals in the positions defined by the hierarchy. I show that the Cohesive Small World model is a solid theoretical framework to define cohesive groups in cooperation networks. The nested structure of $k$-components nicely captures the hierarchy in the patterns of relations that individual contributors establish when working together. This hierarchy, on the one hand, reflects the empirically well established fact that in FOSS projects only a small fraction of the developers account for most of the contributions. And, on the other hand, refutes the naive views of early academic accounts that characterized FOSS projects as a flat hierarchy of peers in which every individual does more or less the same. 

I also show that the position of individual developers in the connectivity hierarchy of the cooperation networks impacts significantly, on the one hand, on the volume of contributions that an individual does to the project. And, on the other hand, the median active life of developers in the project. I argue that the latter is a better way to analyze robustness of FOSS projects than the classical random and targeted attacks that has been used to assess robustness in other kinds of networks.

I argue that the connectivity structure of collaborative communities' cooperation networks can be characterized as an open elite, where the top levels of this hierarchy are filled with new individuals at a high pace. This feature is key for understanding the mechanisms and dynamics that make FOSS communities able to develop long term projects, with high individual turnover, and yet achieve high impact and coherent results as a result of large scale cooperation. I conclude that cooperation in FOSS communities has a structural dimension because membership in cohesive groups that emerge from cooperation networks has an important and statistically significative impact on both the volume of individual contributions, and on the median active life of developers in the projects under analysis.
\end{abstract}

\newpage

\selectlanguage{catalan}%
\begin{abstract}
L'última meitat del segle XX ha estat testimoni d'un canvi fonamental en el procés de producció de coneixement: els descobriments més importants i les innovacions en ciència i tecnologia no són el resultat de la tasca de persones amb molt talent que treballen soles, sinó que són el resultat de processos de cooperació i de treball en equip. El notable augment de l'escala de la cooperació en els processos de producció intensius en coneixement ha renovat l'interès en l'anàlisi dels mecanismes pels quals emergeix i prospera la cooperació a gran escala.

Els dos principals enfocaments teòrics sobre la cooperació són, d'una banda, un enfocament micro que considera que la cooperació com un procés atòmic en el qual l'interès es centra en com es produeix cooperació entre dues persones i, d'altra banda, com un fenomen a nivell macro en el qual el centre de l'anàlisi és el grup com a col·lectivitat. L'objectiu d'aquesta recerca és acostar posicions entre l'enfoc macro i el micro sobre la cooperació tot centrant-se en els mecanismes a nivell meso, que fins fa poc han rebut poca atenció en el debat teòric. El meu argument és que un enfocament de nivell meso ha de centrar-se en la dimensió estructural de la cooperació, és a dir, en els patrons de relacions entre les persones que participen directament en els processos de producció, el que jo anomeno xarxes de cooperació. Aquesta perspectiva mostra que entre les interaccions diàdiques entre els individus, i els grans objectius i valors compartits que guien les grans organitzacions i grups, hi ha subgrups d'individus que tenen un paper fonamental en generar i fomentar la cooperació a gran escala de la que hem estat testimonis en les últimes dècades.

Aquesta recerca es centra en l'estudi de cas de dos projectes de programari lliure (FOSS en anglès) ---el sistema operatiu Debian i el llenguatge de programació Python--- per tal de construir un marc teòric estructural que ens ajudi a explicar i entendre com funciona la cooperació gran escala. En aquesta tesi presento un model de xarxa, que anomeno ``Cohesive Small World'', que es basa en dos models teòrics ben establertes: el model ``Small World'' i el model de cohesió estructural. Proposo que aquests dos models no són mútuament excloents. La família de xarxes que s'ajusten a la intersecció de tots dos models mostren patrons estructurals consistents. Aquests patrons proporcionen els fonaments per al sorgiment de comunitats de col·laboració, com ara projectes de programari lliure, i tenen un paper clau en fomentar la cooperació a gran escala. 

D'una banda, els grups estructuralment cohesius en la jerarquia de connectivitat de les xarxes de cooperació generen confiança i valors compartits entre individus heterogenis perquè els individus inclosos en aquestes estructures poden comparar perspectives independents sobre cadascun dels altres membres de la col·lectivitat a través de múltiples intermediaris, la qual cosa els proporciona múltiples fonts d'informació independents. Per tant, les persones incloses en aquests grups cohesius, tenen una percepció dels altres membres de la xarxa de cooperació amb qui no estan directament connectats que està filtrada per altres membres d'aquests grups cohesius amb qui confien perquè hi estan directament connectades. Aquesta percepció mediada pels grup cohesius genera confiança i valors compartits a escala global. D'altra banda, l'existència de \emph{clusters} locals ---grups de persones que treballen estretament entre elles--- connectats per distàncies relativament curtes amb altres \emph{clusters} de la xarxa de cooperació, permet la cooperació entre individus heterogenis amb interessos comuns i, al mateix temps, fomenta el flux d'informació entre aquests \emph{clusters} que impedeixen que aquests grups de persones que treballen estretament entre elles siguin atrapades en caixes de ressonància formades per col·laboradors afins amb les mateixes idees.

En la part metodològica de la tesi, he desenvolupat heurístiques per a calcular l'estructura de $k$-components de les xarxes de cooperació. Aquestes heurístiques permeten calcular en un temps raonable el valor aproximat de la cohesió dels grups en xarxes de cooperació moderadament grans, juntament amb tota l'estructura jeràrquica dels diferents nivells de connectivitat. En la tesi demostro com aquestes heurístiques poden ser aplicades a xarxes almenys un ordre de magnitud més grans que les que podia assumir l'únic algoritme disponible fins ara. Amb l'ajuda d'aquestes heurístiques poso a prova empíricament el nou model que proposo per tal de millorar la nostra comprensió de com funciona la cooperació en les comunitats de col·laboració. L'anàlisi empírica demostra que el model estructural que proposo en la part teòrica s'ajusta als patrons de cooperació que observem en els projectes de programari lliure analitzats en la part empírica de la tesi.

L'anàlisi empírica d'aquesta tesi explora la dimensió dinàmica de les jerarquies de connectivitat que sorgeixen en les xarxes de cooperació dels projectes de Python i Debian. Defineixo xarxes de cooperació com els patrons de relació entre les persones que participen en els processos productius dels projectes analitzats. L'anàlisi dinàmic que presento no és només una anàlisi longitudinal dels canvis en la jerarquia a través del temps, sinó també una anàlisi del ritme de renovació dels individus en les posicions definides per aquesta jerarquia. Demostro que el model estructural que proposo és un marc teòric sòlid per tal de definir grups cohesius en les xarxes de cooperació. L'estructura d'aquests grups cohesius defineix la jerarquia de connectivitat dels patrons de relacions que estableixen els individuals al treballar conjuntament. Aquesta jerarquia, d'una banda, reflecteix el fet empíricament ben establert que en projectes de programari lliure només una petita part dels participants contribueix la major part de la feina feta en cada projecte. I, d'altra banda, refuta les opinions ingènues dels primers relats acadèmics que caracteritzen els projectes de programari lliure com una jerarquia plana de persones en la qual cada individu fa més o menys el mateix.

L'anàlisi empírica d'aquesta tesi també mostra que la posició dels desenvolupadors individuals en la jerarquia de connectivitat de les xarxes de cooperació impacta significativament, d'una banda, en el volum de les contribucions que cada persona fa al projecte. I, d'altra banda, en el temps de vida mitjana de les persones en el projecte, entesa com el temps que de mitjana una persona és participant activa en el projecte.

Finalment, argumento que l'estructura de connectivitat de xarxes de cooperació de les comunitats de col·laboració pot caracteritzar-se com una elit oberta, on els nivells més alts d'aquesta jerarquia es renoven constantment amb la incorporació de noves persones. Aquesta característica és clau per entendre els mecanismes i dinàmiques que fan que les comunitats de programari lliure siguin capaces de desenvolupar projectes a llarg termini, amb un alt volum de renovació individual, i no obstant això, aconsegueixin uns resultats coherents com a resultat de la cooperació a gran escala. Finalment concloc que la cooperació en les comunitats de programari lliure té una dimensió estructural ja que la pertinença a grups cohesius que sorgeixen en les xarxes de cooperació té un impacte important i estadísticament significatiu tant en el volum de les contribucions individuals com en la vida activa mitjana de les persones que participen en els projectes analitzats en aquesta tesi.
\end{abstract}
\selectlanguage{english}%

