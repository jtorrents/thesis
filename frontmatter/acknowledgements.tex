\chapter*{Acknowledgments}

This thesis is the result of many years of work, many more than the current standards prescribed by the Spanish law that regulates PhD studies. This has been both a blessing and a curse; a blessing because working on this thesis allowed me to explore many theoretical approaches and empirical analysis relevant to the main research question. This exploration expanded my abilities in fields, such as programming and statistical analysis, that were not prominent in my undergraduate and graduate studies. But it has been also a curse because I had to discard a lot of work and interesting approaches explored during these years in order to finish the thesis and obtain a coherent final product.

Many people has helped me, in many ways, during this long journey. First of all I'd like to thank my two co-directors and mentors: Joaquim Sempere and Fabrizio Ferraro. Joaquim Sempere, with his work as Sociological Theory professor during my undergraduate and graduate studies, provided me with the theoretical background, focused specially in the classics of Sociology, in general, and Marx, in particular, needed to tackle big and relevant research questions. This background and his approach of always reading directly the classics with an open mind and an eye on the current problem ---instead of reading works that discuss the work of the classics--- has helped me enormously to deal with the research problems that I had to deal in doing this thesis. During all this time, he has always been available for long theoretical ---and political--- discussions, sometimes only tangentially related to the thesis but always relevant to me. He has also had a lot of patience dealing with my highly irregular work schedules on the thesis, consisting in weeks or months of inactivity followed by frenetic periods of work.

Fabrizio Ferraro, for whom I worked as a research assistant during five years, has been the other big influence and mentor that guided me in the long process of working on this thesis and finishing it. His knowledge of contemporary Sociological theory, and especially, Organization theory, has been key for me as these theoretical approaches were almost unknown to me before working with him. His focus on solid empirical research on his own work has been an example and a guide for my own empirical work presented on this thesis. He has always been open minded regarding the tools used to perform empirical analysis, and gave me the opportunity to spend time learning how to program and how to perform sophisticated statistical and network analysis using Free Software tools. The skills that I developed working with him have been key for many ---if not most--- of the methodological and empirical elements that conform this thesis.

I'd like thank professor Juan Díez Medrano for his encouragement and detailed reading and comments of my work in the early stages of my PhD. I'd like to also thank professor Maite Montagut, my tutor for this thesis, for her support during the last stages of my PhD. I'd like to also thank the members of my thesis committee ---professors María Trinidad Bretones, Josep Lluís C. Bosch, and Lluís Coromina--- for their guidance and supervision of my work. I'd like to also thank professor Pep Rodriguez, the director of the PhD program at University of Barcelona, for his encouragement and interesting discussions about my work in this thesis, and about how statistics and network analysis should be taught in Sociology undergraduate and graduate courses. Finally I'd like to thank Matteo Prato, professor at University of Lugano, and David Stark, professor at Columbia University, for the opportunity to work with them in cutting edge sociological research.

Finishing a big project such as a PhD thesis requires not only academic guidance and encouragement, but also personal and administrative support. In that regard, I'd like to thank professor Montser Ferràs for her work coordinating the Sociology PhD program at University of Barcelona, for her continuous encouragement during the last years of my PhD, and for her help in dealing with the paperwork associated with the many changes that the Sociology PhD program at University of Barcelona had to endure because of the many legislative changes in tertiary education. I'd like to also thank Eloísa Pérez, the director of the ``Oficina de Màsters i Doctorat'' at Faculty of Economics of University of Barcelona, and all the staff ---Elisa Gutiérrez, Àngels Pascual, Eva Sáez, Maria Dolors Vázquez--- for their support dealing with the tons of paperwork needed for keeping up with the many administrative reforms.

Finally I'd like to thank my parents, Montserrat Vivó and Josep Mª Torrents, for their love and continuous encouragement and support during the ups, downs, and many turns of my university student life, which started as a Biology undergraduate back in 1996. And last, but not least, thanks to Laura Rafecas for being always there, and for her jokes, sparkled with a bit of skepticism, about the time that would take me to finish this thesis.

\newpage
