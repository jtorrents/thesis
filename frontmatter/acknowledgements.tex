\chapter*{Acknowledgments}

This thesis is the result of many years of work, many more than the current standards prescribed by the Spanish law that regulates PhD studies. This has been both a blessing and a curse; a blessing because working on this thesis allowed me to explore many theoretical approaches and empirical analysis relevant to the main research question. This exploration expanded my abilities in fields, such as programming and statistical analysis, that were not prominent in my undergraduate and graduate studies. But it has been also a curse because I had to discard a lot of work and interesting approaches explored during these years in order to finish the thesis and obtain a coherent final product.

Many people has helped me, in many ways, during this long journey. First of all I'd like to thank my two co-directors and mentors Joaquim Sempere and Fabrizio Ferraro. Joaquim Sempere, with his work as Sociological Theory professor during my undergraduate and graduate studies, provided me with the theoretical background, focused specially in the classics of Sociology, in general, and Marx, in particular, needed to tackle big and relevant research questions. This background and his approach of always reading directly the classics with an open mind and an eye on the current problem ---instead of reading works that discuss the work of the classics--- has helped me enormously to deal with the research problems that I had to deal in doing this thesis. During all this time, he has always been available for long theoretical ---and political--- discussions, sometimes only tangentially related to the thesis but always relevant to me. He has also had a lot of patience dealing with my highly irregular work schedules on the thesis, consisting in weeks or months of inactivity followed by frenetic periods.

Fabrizio Ferraro, for whom I worked as a research assistant during five years, has been the other big influence and mentor that guided me in the long process of working on this thesis. His knowledge of contemporary Sociological theory, and especially, Organization theory, has been key for me as these theoretical approaches were almost unknown to me before working with him. His focus on solid empirical research on his own work has been an example and a guide for my own empirical work presented on this thesis. He has always been open minded regarding the tools to perform the empirical analysis and gave me the opportunity to spend time in learning how to program and to perform sophisticated statistical andi network analysis using Free Software tools. The skills that I developed working with him have been key for many ---if not most--- of the methodological and empirical elements that conform this thesis.

I'd like to also thank 

\newpage
